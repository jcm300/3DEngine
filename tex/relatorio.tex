\documentclass{article}
\usepackage[utf8]{inputenc}
\usepackage{graphicx}
\usepackage{float}

\begin{document}

\title{\vspace*{\fill}3D Engine}
\author{João Vieira (a78468) \and José Martins (a78821) \and Miguel Quaresma (a77049) \and Simão Barbosa (a77689)}
\date{%
    Universidade do Minho\\
    Computação Gráfica\\[2ex]%
    \today\vspace*{\fill}
}
\maketitle

\newpage

\tableofcontents

\newpage

\section{Introdução}
Este documento serve de documentação à implementação do generator bem como engine desenvolvido com o intuito de no futuro ser possivel a construção de sistemas planetários virtualmente e graficamente.

\section{Generator}

\subsection{Plano (Plane)}

\subsection{Caixa (Box)}

\subsection{Esfera (Sphere)}

\subsection{Cone (Cone)}

\section{Engine}
Em termos de bibliotecas no engine, são utilizadas as seguintes:
\begin{itemize}
    \item Libxml2: parser XML utilizado devido à sua alta utilização bem como simplicidade
    \item OpenGL: Usado para desenhar sendo a sua utilizão justificada devido a ser open-source e devido à presença de boa documentação
    \item Glut: Biblioteca de "colagem" entre OpenGL e SO
    \item Glew: necessário para a utilização de VBO's
\end{itemize}
Portanto, com recurso ao mesmo (libxml2) obtivemos do ficheiro XML e pela ordem apresentada no mesmo, os ficheiros em que cada um possui pontos referentes a um poliedro/poligono, sendo este ficheiros possiveis de ser gerados pelo generator atrás referido. 
Para guardar estes pontos é utilizado uma estrutura (Points) que é uma lista ligada, sendo cada elemento constituido por:
\begin{itemize}
    \item Um array com todas as coordenadas dos pontos disponibilizados por determinado ficheiro
    \item O numero de pontos que essas coordenadas "geram"
    \item Um apontador para a próxima estrutura (ficheiro seguinte)
\end{itemize}
De modo a desenhar os pontos correspondentes aos presentes nos ficheiros e agora também nos arrays, é usado VBO's facilitando e melhorando a performance. As figuras são desenhadas por um conjunto de triangulos, e como tal a cada 3 pontos é desenhado um. 
Foi também implementado, recorrendo a coordenadas polares, a posibilidade de mover a camera:
\begin{itemize}
    \item Rodar horizontalmente à esquerda -> tecla seta esquerda
    \item rodar horizontalmente à direita -> tecla seta direita
    \item rodar verticalmente para cima -> tecla seta cima
    \item rodar verticalmente para baixo -> tecla seta baixo
    \item aproximar -> tecla 'w'
    \item afastar -> tecla 's'
\end{itemize}
Para além disso, carregando na tecla 'm' é alterado consecutivamente o modo de display dos poligonos, alternando entre FILL (a cheio), LINE (em linhas) e POINT (apenas pontos).

\section{Glossário}

\subsection{Coordenadas Polares}

\subsection{VBO's}

\section{Conclusão}

\end{document}
